Over the last 50 years, robotic and automation systems have transformed our world and greatly enhanced the quality of our daily life. With the development of science and technology, many intelligent systems which integrate machines, electronics, automatic control and information technologies have emerged. Such systems can accomplish numerous tasks originally performed by humans and often prove superior in terms of precision, speed and cost. They can replace humans in the tasks that require repetitive and monotonous operations. For example, in the automotive industry, robotic and automation systems have been widely used for manufacturing, assembling and painting. From the point of engineering, they have optimized the process of productivity and improved the productive efficiency, thus greatly increasing the speed of industrialization. 

Robotic and automation systems also contribute to scientific research, especially in some situation that requires to conduct experiments in dangerous environments (e.g, nuclear factory) which are hazardous to human beings or some operating environments that may be beyond humans' capabilities of reach (e.g., other planets). In 2003, two famous robots---\textit{Spirit} and~\textit{Opportunity} were sent to Mars by NASA to explore the surface and geology of this planet~\cite{Grotzinger:Sci:2014}. With the help of intelligent/robotic systems, researchers can collect data much faster than ever before. For instance, the high-throughput screening (HTS) systems~\cite{Hertzberg2000}, which are widely used in the drug discovery and chemistry, allow the researchers to conduct millions of experiments and collect data in a very short time. Such systems consist of several components, including data analysis software, robotics, liquid handling device, etc. Besides data collection, these systems can also analyze the data automatically using intelligent software, which provides an ideal tool for data analysis in scientific research and frees researchers from the tedious and monotonous process of data analysis if done manually. This accelerates the development of scientific research to a great extent. 
%In~\cite{Eisenstein_2006}, Eisenstein argues that, ``soon, if a scientist does not understand some statistics or rudimentary data-handling technologies, he or she may not be considered to be a pure researcher and thus will simply become a dinosaur.'' 
%In data-driven science~\cite {Golub_2010}, where researchers have to deal with a huge amount of data collected in the experiments, 

Intelligent systems also play an important role in~\textit{ethology}, which is the scientific study of animal behavior~\cite{Bolhuis_2004}. Ethology is pursued not only because it is a subject of interest in itself, but also because the knowledge gained from it has several practical applications. For instance, models of animal decision-making processes can be used to predict their behavior in novel environments, which can help in making ecological conservation policy~\cite{Sutherland1998}. Knowledge about animal behaviors has also been applied for solving computational problems~\citep{Floreano2008}, and for constructing biologically-inspired robotic agents~\citep{Meyer2008}. There are four types of questions to be investigated in ethology: questions concerning causes, functions, development and evolution~\cite{Bolhuis_2004}. Causes refer to the mechanisms of animals that are innate as well as the external/internal stimuli that affect such behavior. Functions concern what is the purpose of this behavior, for example, foraging or matting. The development of animal behavior deals with how animals learn such behavior during their whole life as well as how such behavior is affected by experience, while evolution relates to how the behavior changes over generations in the course of natural evolution. Over centuries, these four questions have been investigated by ethologists either in a well-controlled laboratory or an outdoor environment. Before the emergence of computers, to investigate the animal behaviors, the ethologists need to observe the animals, analyze data by hand. They also need to learn how to control the environmental conditions in a meaningful way to extract most of the information from the animals under investigation. However, such process of analysis sometimes is very time-consuming and tedious. With the help of intelligent and robotic systems, nowadays researchers can conduct experiments much more efficient.

%To use such bio-inspired methods to solve engineering problem, researchers need to have a good understand of the animal behaviors. However, the problem is that researchers are still in lack of a detailed understanding of the internal mechanisms underlying animal behaviors, which greatly limits the development of bi-inspired artificial intelligence and bio-inspired robotics. In order to make robots or computers mimic the highly intelligent behavior, computer scientists or roboticists should first build a model of target animal behavior based on knowledge in the field of ethology. 

%Studying animal behavior has several scientific advantages. It can help ethologists better understand the internal mechanism or causes, functions, development as well as natural evolution of such behaviors. From the point of engineering, researchers in robotics area, for instance, are more interested in getting inspiration from animal behavior and building robots that mimic the intelligent behavior to solve complex tasks in reality.

\section{Motivation}

Although intelligent and automation systems have played an significant role in scientific research, they are often secondary. In most of the cases, these systems are just doing mechanical and repetitive work. The question is whether we can build a machine/system that can dominate the whole process of scientific investigation and automatically analyze experimental data, search for correlation between different elements, and generate new hypotheses. In other words, can we build a system which is able to automatically conduct scientific research without (or with minimal) human intervention? Recently, the emergence of ``robot scientists'' shows that such systems are within reach~\cite{King_2009, Evans_2010, Waltz2010}. Following this motivation, this thesis aims to pave the way for further development in science automation~\cite{Evans_2010}, especially in the area of ethology. In particular, we present a new system identification method---\textit{Turing Learning}, that can automatically learn/model agent behavior\footnote{Since the behaviors under investigation in this thesis are simulated using computer simulation or physical robots, throughout the thesis, unless otherwise stated, we used the term agent behavior.} with minimal human intervention. The ultimate goal of this thesis is to contribute to the study of animal behavior through developing an automatic system identification system.

System identification is a process of modeling natural or artificial systems through observed input and output data. It has drawn a large interest among researchers for decades~\cite{Ljung2010, Billings2013}. One application of system identification is the reverse engineering of agent behavior (biological organisms or artificial agents). Many studies have investigated how to deduce rules of agent behavior using system identification techniques based on macro models~\cite{Shandelle2010}. When investigating the interaction within a group of agents and between the agents and environments, agent-based models~\cite{Bonabeau2002} can provide a good representation of such behaviors. An agent-based model is a type of micro model, in which the individual rules are modeled, and the global behavior emerging from interaction is used for refining the models.  Evolutionary computation which draws inspiration from biological evolution (will be introduced in Section~\ref{sec:evolutionary_computation}) has proven to be a powerful method to automate the generation of models, especially for behaviors that are hard to formulate~\cite{Bongard2005_tevc,Bongard2007PNAS, Ruxton2008}. Evolutionary computation provides a potential realization for automation science, as models evolve in an autonomous manner. It is the main technique that is investigated in this thesis for performing system identification.

A limitation of current system identification methods is that they rely on predefined metrics, such as the square error, to measure the difference between the output of the models and that of the system under investigation. Model optimization then proceeds by minimizing the measured differences. However, for complex systems, defining a metric can be non-trivial and case-dependent. It may require significant prior information about the systems. Moreover, an unsuitable metric may not distinguish well between good and bad models, or even bias the identification process. This thesis overcomes these problems by introducing a system identification method that does not rely on predefined metrics.

%, which are executed by replicas/animats 
%The first population contains \textit{models}. The second population contains \textit{classifiers}. These two populations coevolve competitively. The fitness of the classifiers depends solely on their ability to distinguish the behavior of the replicas from the behavior of the agents under investigation.  The fitness of the models depends solely on their ability to mislead the classifiers into making the wrong judgment, that is, classifying them as the agent. In this way, the approach does not require any predefined metrics to quantify the difference between the behaviors of models and agents. 

\section{Problem Statement}

In this thesis, we applied our method to infer agent behaviors, ranging from swarm behaviors to deterministic/stochastic behaviors of a single agent, using three case studies. The agent to be studied is put in an environment. Its behavior depends on interaction with the environment and with other agents in a group (if any). The system identification task is to learn the observed behavior, in other words, the agent's behavioral rules through observation. In general, one could monitor a range of the agent's states including body temperature, blood sugar level, morphology, etc. In this project, the machine will observe the animal's motion which is the simplest case, and assumes that the intelligent system is possible to track the position and orientation of the agent at discrete steps in time. 

The first case study we investigated here is inferring/modeling swarm behaviors, which are emergent behaviors that arise from the interactions of numerous simple individuals~\cite{Camazine2001}. Learning about behaviors that are exhibited in a collective manner is particularly challenging, as the individuals not only interact with the environment but also with each other. Typically their motion appears stochastic and is hard to predict~\cite{Dirk2011}. For instance, given a swarm of simulated fish, one would have to evaluate how close its behavior is to that of a real fish swarm, or how close the individual behavior of a simulated fish is to that of a real fish. Characterizing the behavior at the level of the swarm (that is, an emergent behavior) is challenging~\cite{Harvey:SI:2015}. It may require domain-specific knowledge and not discriminate among alternative individual rules that exhibit similar collective dynamics~\cite{Weitz2012}. Characterizing behavior at the level of individuals is also difficult, as even the same individual fish in the swarm is likely to exhibit a fundamentally different trajectory every time it is being looked at. Therefore, in this case study, we investigate how \textit{Turing Learning} can be used to automatically infer the individual rules of a group of homogeneous agents only through observation.

The second case study is about inferring the deterministic behaviors of a single agent. In particular, we investigate how the agent responds to its environmental stimulus. The behavior of the agent under investigation is deterministic and depends solely on the environmental stimulus. However, in order to automatically extract all the agent's behavioral repertoire and infer their behavioral rules, the machine needs to construct complex patterns of stimulus that help reinforce the learning process. In our method, the machine has full control over the environmental stimulus that the agent responds to, and at the same time observes the agent's actions. Typically we investigate how~\textit{Turing Learning} can be used for learning such deterministic behaviors which has low observability.

In the third case study, the method is applied to learn stochastic behaviors of a single agent. In this case, the agent still responds to the environmental stimulus; however, its behavior is not only determined by the environmental stimulus. In other words, constructing a fixed sequence of stimulus may not trigger all the agent's behavioral repertoire as in the case of investigating deterministic behaviors. The machine needs to interact with the agent during the experimental process and makes decision (e.g., how to change/control the environmental stimulus) based on the agent's current states to trigger its hidden behavior. This intelligent behavior is widely observed in the experiments involving human beings. Here, whether a machine could exhibit such intelligent behavior and infer the agent's behavioral rules is challenging. It is hence investigated in this thesis as a case study.  

%Whether this active learning proves to be superior to passive learning which is only based on observation is a question investigated in this thesis. 
%,and assumes that the intelligent system is possible to track the position of the animal at discrete steps in time
%The system is able to learn the observed behavior, in other words, the animal's actions in response to the different stimuli and combinations thereof. It constructs on-the-fly patterns of stimuli that help reinforce the learning process, so that the system can automatically extract the model of animal behavior. This interactive approach proves superior to learning through passive observation. 

\section{Contributions}

\begin{itemize}

\item A novel system identification approach---\textit{Turing Learning} which allows a machine to infer agent behaviors in an autonomous manner. \textit{Turing Learning} uses a coevolutionary algorithm (which will be introduced in Section~\ref{sec:evolutionary_computation}) comprised of two populations. A population of candidate models competitively coevolves with a population of classifiers. The classifiers observe the models and agents. The fitness of the classifiers depends solely on their ability to discriminate between them. Conversely, the fitness of the models depends solely on their ability to `trick' the classifiers into categorizing them as agents. Unlike other system identification methods, \textit{Turing Learning} does not rely on predefined metrics to gauge the difference between the behaviors of agents and models. 
%This eliminates potential bias that predefined metrics may have on the solutions obtained. 
%This method is inspired by the Turing test~\cite{Turing_1950, Harnad2000}, which machines can pass if behaving indistinguishably from humans. Similarly, the models, which evolve, can pass the tests by the coevolving classifiers if behaving indistinguishably from the agents.  

\item Applying \textit{Turing Learning} to successfully infer the behavioral rules of a group of homogeneous agents. Both the model parameters, which were automatically inferred, and emergent global behaviors closely matched those of the original swarm system. 

\item A systematic investigation of the evolved classifiers in \textit{Turing Learning}. We constructed a robust classifier system that, given an individual's motion data, can tell whether the individual is an original agent or not. Such classifier system could be effective in detecting abnormal behavior, for example, when faults occur in some members of the swarm.

\item A realization of \textit{Turing Learning} to automatically perform system identification directly through observation of swarms of physical robots. The results in physical experiments showed good correspondence to those obtained in simulation. 

\item Extending \textit{Turing Learning} to automatically infer deterministic behavior of a single agent in simulation by interacting with it, rather than simply observing its behavior in a passive manner. This interactive approach proves superior to learning through passive observation. 
%This interaction can help the machine to extract all of the agent's behavioral dynamics through outputting a complex sequence of input which is difficult to generate by random input. 

\item Extending \textit{Turing Learning} to automatically infer the stochastic behaviors of a single agent in simulation through controlled interaction. The model parameters are successfully identified. The evolved classifiers show clear interaction with the agent through changing the environmental conditions (stimuli) based on the behavioral dynamics of the agent during the experimental process. The results are shown to be better than those obtained using metric-based system identification methods. 
% In this case, outputting a fixed sequence of input is not sufficient for the machine to extract all the information from the agent due to its stochastic features. 
\end{itemize}

\section{Publications}

This thesis presents the author's own work. Some parts of the thesis have been published as original contributions to the scientific area. A preliminary work of Chapter~\ref{ch:swarm_simulation} was orally presented in a conference by the author:
\begin{itemize}
%
\item \textbf{W. Li}, M. Gauci and R. Gro{\ss}, ``Coevolutionary learning of swarm behaviors without metrics,'' \textit{Proceedings of 2014 Genetic and Evolutionary Computation Conference (GECCO 2014)}. ACM Press, Vancouver, Canada, 2014, pp. 201--208.
%
\end{itemize}

A preliminary work of Chapter~\ref{ch:interaction} was orally presented in a conference by the author of this thesis:
\begin{itemize}
%
\item \textbf{W. Li}, M. Gauci and R. Gro{\ss}, ``A coevolutionary approach to learn animal behavior through controlled interaction,'' \textit{Proceedings of 2013 Genetic and Evolutionary Computation Conference (GECCO 2013)}. ACM Press, Amsterdam, Netherlands, 2013, pp. 223--230.
%
\end{itemize}

A part of Chapters~\ref{ch:swarm_simulation} and~\ref{ch:swarm_physical_implementation} has been written as a paper and submitted to the following journal:
\begin{itemize}
%
\item \textbf{W. Li}, M. Gauci, J.Chen and R. Gro{\ss}, ``Reverse Engineering Swarm Behaviors Through Turing Learning,'' \textit{IEEE Transactions on Evolutionary Computation}, under review.
%
\end{itemize}

Apart from the work presented in this thesis, the author has also contributed to some other projects. This led to the following publications:

\begin{itemize}
%
\item M. Gauci, J. Chen, \textbf{W. Li}, T. J. Dodd, and R. Gro{\ss}, ``Self-organized aggregation without computation,'' \textit{The International Journal of Robotics Research}, vol. 33, no. 8, pp. 1145--1161, 2014.
%
\item J. Chen, M. Gauci, \textbf{W. Li}, A. Kolling and R. Gro{\ss}, ``Occlusion-based cooperative transport with a swarm of miniature mobile robots.''\textit{ IEEE Transactions on Robotics}, vol.31, no.2, pp. 307--321, 2015.
%
\item M. Gauci, J. Chen, \textbf{W. Li}, T. J. Dodd, and R. Gro{\ss}, ``Clustering objects with robots that do not compute,'' in \textit{Proceedings of the 13${\textrm{th}}$ International Conference on Autonomous Agents and Multiagent Systems (AAMAS 2014)}. IFAAMAS Press, Paris, France, 2014, pp. 421--428.
%
\end{itemize}

During his PhD studies, the author has also been a Marie Curie Research Fellow with the Department of Mechanical Engineering, University of Western Ontario, Canada, where he contributed to the project of Mechanical Cognitivization. This led to the following publications:

\begin{itemize}
%\item G. Avigad, \textbf{W. Li}, A. Weiss, ``Enhancing Robustness through Mechanical Cognitivization'' \textit{International Journal on Advances in Intelligent Systems}, vol.7, no.3, pp. 652--661, 2014.
%
\item G. Avigad, \textbf{W. Li}, A. Weiss, ``Mechanical Cognitivization: A Kinematic System Proof of Concept'' \textit{Adaptive Behavior}, vol.23, no.3, pp. 155--170, 2015.
%
\end{itemize}

\section{Thesis Outline}

This thesis is structured as follows:

\begin{itemize}
\item Chapter~\ref{ch:literature_review} describes the background of the thesis as well as the related work presented in the literature. 

\item Chapter~\ref{ch:swarm_simulation} introduces the metric-free system identification method---\textit{Turing Learning}. It is applied to learn two swarm behaviors (self-organized aggregation~\cite{Gauci2014_ijrr} and self-organized object clustering~\cite{Melvin2014_aamas}) through observation. This chapter is organized as follows. Section~\ref{sec:methodology_swarm_simulation} describes the implementation of~\textit{Turing Learning} (Section~\ref{sec:turing_learning_swarm_simulation}) and the two swarm behaviors (Section~\ref{sec:case_studies_swarm_simulation}) investigated in this thesis. Section~\ref{sec:simulation_platform_setups} introduces the simulation platform (Section~\ref{sec:platform_swarm_simulation}) and simulation setups (Section~\ref{sec:setup_swarm_simulation}) for performing coevolution runs. Section~\ref{sec:results_swarm_simulation} presents the results obtained from the two case studies. Section~\ref{sec:analysis_evolved_models_swarm_simulation} systematically analyzes the evolution of models, through objectively measuring the evolved models in terms of their local and global behaviors. Section~\ref{sec:coevolutionary_dynamics_simulation_swarm_simulation} investigates the coevolutionary dynamics. Section~\ref{sec:analysis_evolved_classifiers_swarm_simulation} systematically investigates the evolution of classifiers, showing how to construct a robust classifier system to potentially detect abnormal behaviors in the swarm. Section~\ref{sec:observing_a_subset_agents_swarm_simulation} studies the effect of observing only a subset of agents in the swarm and the results obtained. Section~\ref{sec:evolving_control_and_morphology_swarm_simulation} presents a study where an aspect of the agents' morphology (their field of view) and brain (controller) are inferred simultaneously. Section~\ref{sec:evolving_other_behaviors_swarm_simulation} shows the results of using~\textit{Turing Learning} to learn other swarm behaviors.. Section~\ref{sec:noise_study_swarm_simulation} presents a noise study. Section~\ref{sec:summary_simulation_swarm} summaries the findings in this chapter.

\item Chapter~\ref{ch:swarm_physical_implementation} presents a real-world validation of~\textit{Turing Learning} to infer the behaviors of a swarm of physical robots. Section~\ref{sec:experimental_setup_swarm_physical} introduces the physical platform, which includes the robot arena (Section~\ref{sec:robot_arena_physical_swarm}), the robot platform and sensors implementation (Section~\ref{sec:robot_platform_sensor_implementation}). Section~\ref{motion_capture_and_video_processing_swarm_physical} details the tracking system, including motion capture (Section~\ref{sec:motion_capture_swarm_physical}) and video processing (Section~\ref{sec:video_processing_physical_swarm}). Section~\ref{sec:coevolution_physical_robots_swarm_physical} describes the programs executed by each component (machine, agent and replica) during the coevolutionary learning process. Section~\ref{sec:experimental_setup_swarm_physical} describes the experimental setup. Section~\ref{sec:experimental_results_swarm_physical} discusses the results obtained. This includes the analysis of the evolved models (Section~\ref{sec:analysis_evolved_model_physical_swarm}) and the analysis of the evolved classifiers (Section~\ref{sec:analysis_evolved_classifier_physical_swarm}). Section~\ref{sec:analysis_algorithm} analyzes the sensitivity of~\textit{Turing Learning} for individual failure during the experimental process. Section~\ref{sec:summary_swarm_physical} summaries the results obtained and discusses the findings in this chapter.

\item Chapter~\ref{ch:interaction} presents two case studies to learn the deterministic and stochastic behaviors of a single agent, respectively. In these case studies, the system not only observes the behavior of the agent but also interacts with the agent through changing the stimulus that can influence the agent's behavior. Section~\ref{sec:deterministic_behavior_interaction} describes the deterministic behaviors under investigation. Section~\ref{sec:results_interaction_deterministic} discusses the results obtained for learning the deterministic behaviors. Section~\ref{sec:stochastic_behavior_interaction} presents the stochastic behaviors under investigation. Sections~\ref{sec:results_interaction_stochastic_2states} and ~\ref{sec:results_interaction_stochastic_3states} discuss the results obtained for learning stochastic behaviors. Section~\ref{sec:summary_interaction} summaries the findings in this chapter.

\item Chapter~\ref{ch:conclusion} concludes the thesis and discusses the future work. 

\end{itemize}
