This thesis concerns the automated reverse engineering of agent behaviors. It proposes a metric-free coevolutionary approach---\textit{Turing Learning}, which allows a machine to infer the behaviors of agents (simulated or physical ones), in a fully automated way. 

\textit{Turing Learning} consists of two populations. A population of models competitively coevolves with a population of classifiers. The classifiers observe the models and agents. The fitness of the classifiers depends solely on their ability to distinguish between them. The models, on the other hand, are evolved to mimic the behavior of the agents and mislead the judgment of the classifiers. The fitness of the models depends solely on their ability to `trick' the classifiers into categorizing them as agents. Unlike other methods for system identification, \textit{Turing Learning} does not require any predefined metrics to quantitatively measure the difference between the models and agents.

The merits of \textit{Turing Learning} are demonstrated using three case studies. In the first case study, a machine automatically infers the behavioral rules of a group of homogeneous agents only through observation. A replica, which resembles the agents under investigation in terms of behavioral capabilities, is mixed into the group. The models are to be executed on the replica. The classifiers observe the motion of each individual in the swarm for a fixed time. Based on the individual's motion data, a classifier makes a judgment indicating whether the individual is believed to be an agent or the replica. The classifier gets a reward if and only if it makes the correct judgment. In the second and third case studies, \textit{Turing Learning} is applied to infer deterministic and stochastic behaviors of a single agent through controlled interaction, respectively. In particular, the machine is able to modify the environmental stimuli (to which the agent responds) and thereby interact with the agent. This allows the machine to reveal all of the agent's entire behavioral repertoire and help reinforce the learning process. This interactive approach proves superior to learning only through observation. 

%The results suggest that system identification can be realized without predefined metrics. A machine could perform 
