This thesis is about automatically learning agent behaviors through machine intelligence. In particular, we have investigated a new metric-free coevolutionary approach---\textit{Turing learning}, which allows a machine to infer the agent behaviors (simulated using computer simulation and physical robotic system) in a fully automatic way. The ultimate goal is to learn animal behavior with little human intervention and pave the way for science automation. In our approach, a population of candidate models competitively coevolves with a population of classifiers. The fitness of the classifiers depends solely on their ability to discriminate between the models and agents, based on their observed motion. Conversely, the fitness of the models depends solely on their ability to ‘trick’ the classifiers into categorizing them as agents. As such, the approach does not require any predefined metrics to quantify the difference between the behaviors of models and agents.

The merits of the \textit{Turing learning} method were demonstrated using three case studies. In the first case study, the machine can infer the rules of interaction between a group of homogeneous agents through observation. A replica, which resembles the agents under investigation in terms of appearance and behavioral capabilities, is mixed into the group. The models are to be executed on the replica. The classifiers observe the motion of each individual in the swarm for a fixed time interval. Based solely on the individual's motion data, the classifiers each output a Boolean value indicating whether the individual is believed to be an agent or the replica. The models, on the other hand, are evolved to mimic the behavior of the agents and mislead the judgment of the classifiers. In the second and third case studies, the \textit{Turing learning} method is applied to learn deterministic and stochastic behaviors of a single agent through controlled interaction, respectively. In particular, the machine is able to modify the environmental stimuli (in which the agent responses to) and interact with the agent, in order to reveal its entire behavioral repertoire. This interactive approach proves superior to learning through passive observation. 
